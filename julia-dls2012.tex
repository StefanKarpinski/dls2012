\documentclass[9pt]{sigplanconf}

% VBS: Change back to 9pt for paper submission. 12 pages max.

% The following \documentclass options may be useful:
%
% 10pt          To set in 10-point type instead of 9-point.
% 11pt          To set in 11-point type instead of 9-point.
% authoryear    To obtain author/year citation style instead of numeric.

\usepackage{amsmath}
\usepackage{algorithm}
\usepackage{algorithmic}

\newcommand{\Matlab}{MATLAB\textsuperscript{\tiny\textregistered}}

\begin{document}

\conferenceinfo{Dynamic Languages Symposium 2012}
               {October 22, 2012, Tucson, USA.}
\copyrightyear{2012}
% \copyrightdata{[to be supplied]}

\titlebanner{Julia}      % These are ignored unless
\preprintfooter{Julia}   % 'preprint' option specified.

\title{Julia: a fast dynamic language for technical computing}

\authorinfo{Jeff Bezanson}{MIT}{bezanson@mit.edu}
\authorinfo{Stefan Karpinski}{MIT}{stefan@karpinski.org}
\authorinfo{Viral B. Shah}{}{viral@mayin.org}
\authorinfo{Alan Edelman}{MIT}{edelman@math.mit.edu}

\maketitle
\begin{abstract}
  Dynamic languages have become popular for scientific computing. They
  are generally considered highly productive, but lacking in performance.
  This paper presents Julia, a new dynamic language for technical
  computing, designed for performance from the beginning by adapting and
  extending modern programming language techniques. A design based on
  generic functions and a rich type system simultaneously enables an
  expressive programming model and successful type inference, leading to
  good performance for a wide range of programs. This makes it possible
  for much of Julia's library to be written in Julia itself, while also
  incorporating best-of-breed C and Fortran libraries.
\end{abstract}

\category{D.3.2}{Programming Languages}{Very high-level languages}

\terms
Programming Language, Technical computing, Scientific computing, High
performance computing

\keywords
Programming Language, Technical computing, Scientific computing, High
performance computing

\section{Introduction}

Convenience is winning. Despite continued advances in compiler technology
and execution frameworks for high-performance computing, programmers
continue to prefer high-level dynamic languages for algorithm development
and data analysis in applied math, engineering, and the sciences. Systems
such as \Matlab, R\cite{Rlang}, SciPy\cite{numpy}, Octave \cite{Octave},
and SciLab \cite{scilab} have greatly increased productivity, but are
often found lacking in performance.

The result is a two-tiered software world, where C and Fortran are
used for key libraries and production code, while high-level languages
are used for interaction and scripting workflow.  A new
approach to dynamic language design can change this situation,
providing productivity and performance in one
package. We should embrace the emerging preference for ``scripting''
style languages, and ask how these systems can better provide for the
future of technical computing.

A technical computing language balances the preferences of the end
user, the library developer, and the language designer.
The ``two-tier'' archiecture, though successful for some applications,
takes its toll on each category of programmer.
Users of such
languages overwhelmingly prefer features such as interactivity,
scripting, and type-free programming.
To obtain performance, users face a few unappealing options.
One option is to employ aggressive vectorization, often at the cost of making
code difficult to read, and often increasing memory use.
Another option is to rewrite code in C or Fortran, which many users
are unwilling to learn. Programming in two
languages can also be more complex than using either language by
itself, due to interfacing issues such as converting between type
domains and handling memory reclamation.  These interfacing issues may
also add overhead when calling between layers.

Library developers are happy to do whatever it takes to get
performance --- type annotations, parallelization, or using multiple
languages.
However, rewriting computational kernels in a different
language can put them at a disadvantage, since the resulting
code may not be generic, or requires scripts to
generate code for multiple input types. Distribution of
such libraries is difficult, since they have to be compiled for
various combinations of operating systems and architectures.

From a language designer's point of view, programming in multiple
languages makes it difficult to perform whole-program optimizations. It
is also difficult to perform domain-specific optimizations of C code.
% ...

Fortunately, there has been significant progress in improving the
performance of dynamic languages. Projects like the Python compiler
framework PyPy \cite{pypyjit} have been fairly successful. Similar
efforts exist for languages from LISP onward. The common feature of
all such projects is that they seek to add performance to an existing
language. This is obviously useful, since a large body of existing code
can benefit. But while
promising, these efforts have yet to significantly decrease the need for
the ``two-tier'' approach in practice.
Julia is designed for
performance from the beginning, and we feel this seemingly-subtle
difference turns out to be crucial.

%In our design, the compiler machinery that provides performance
%is also available for extra expressivity in programs.

{\it Built-in} performance means that the compiler's type machinery is
also available within the language, adding expressiveness. This, in
turn, allows more functionality to be implemented in libraries. Many
of the key differences between languages used by different disciplines
(e.g. R for statistics) could be expressed in libraries, but are
instead either part of the language core, or implemented in C where
they are more difficult to modify or extend. When optimizers for these
languages are developed, knowledge of key library functions often must
be encoded into the compiler. Often, languages focus on one particular
type (eg. {\it double precision}. It is cumbersome to write codes that
use floating point numbers with different precisions, or other
built-in types such as integers and strings. Codes that use
user-defined types end up being extremely slow, that users tend to
avoid defining their own types altogether.

%Even Common LISP, for which there are several
%highly-optimizing compilers, specifies arithmetic in the language, and
%yet users do not all agree on how arithmetic should behave. Some
%users require specialized types such as fixed-point numbers or intervals,
%or support for ``missing data'' values as in R.

Julia has the potential to solve this problem by providing
infrastructure that can be shared across domains, without sacrificing
the ease and immediacy of current popular systems.  We take advantage
of, and validate, this infrastructure by writing Julia's standard
library in the language itself, which (1) makes the code more generic
and increases our productivity, (2) allows inlining library code into
user code and vice-versa, and (3) enables direct type analysis of the
library instead of requiring knowledge of library functions to be
built in to the compiler. New users are able to read the standard
library code, and modify it, imitate it, or extend it for their own
purposes.

%Ultimately, performance is about flexibility, not just getting answers
%faster.

%% Ultimately, performance is about more than getting
%% an answer faster; it is about expressivity and flexibility. Our core
%% strategy for achieving this is to employ a sophisticated type system that can
%% nevertheless be ignored by users who aren't interested in it. The type
%% system becomes, in a sense, an optional tool for library writers.

%% often languages start with a performance or parallelization goal and work
%% from there. we start from the opposite direction, designing for maximum
%% flexibility and ease-of-use, and betting that this power can be leveraged to
%% meet increasingly ambitious performance goals. One area where a flexible
%% high-level language can potentially help performance is custom code
%% generation. Often hand-written C code is impractical or insufficient for
%% obtaining the highest performance. Julia's type inference and JIT compiler
%% make it easier to generate efficient code at a more abstract, symbolic level.

%%% NOTE: nominated for deletion by JWB
%% Many of the ideas explored here are not exclusively applicable to technical
%% computing, but we have chosen to target that application area for several
%% reasons. First, technical computing has unique concerns that can be
%% especially awkward or inefficient to handle in existing dynamic languages.
%% Examples include the need for a wide variety of numeric types, and the need for
%% efficient arrays of those types. Second, the performance
%% of high-level technical computing languages has begun to seriously lag behind
%% that of more mainstream languages (notably JavaScript), creating a present
%% need for attempts to improve the situation.
%% General-purpose languages like Java or perhaps even JavaScript could
%% be used for technical computing, but we feel the community will continue to
%% prefer environments that cater to its syntactic needs, and are able to
%% prioritize issues of numerical accuracy and performance.

%% my parallelism principles:
%% 1. People won't use a language *just* to get parallelism. They will
%% live with their favorite language's parallel extensions (ipython,
%% cilk, PCT, etc.)
%% 2. Making parallelism implicit in evaluation semantics is not the way
%% to get effective parallelism. I think I heard Arvind say they tried it
%% with parallel Haskell for 10 years, and it didn't work.
%% 3. If a language isn't as easy as \Matlab or R, people will keep using
%% \Matlab or R.

%\section{The Essence of Julia}

Julia's primary means of abstraction is dynamic multiple dispatch.
Much of a language consists of mechanisms for selecting
code to run in different situations --- from method selection to
instruction selection. We use only dynamic multiple dispatch for this
purpose, which is possible through sufficiently expressive
dispatch rules. To add usability to this flexibility,
types can generally be ignored when not used to specify dispatch behavior.

Types may optionally be used to make declarations, which are considered by
the compiler and checked at run time when necessary. However, we do not
require declarations for performance. To achieve this, Julia's compiler
automatically specializes methods for types encountered at run time
(or at compile time, to the extent types are known then). Effectively,
every method is a template (in the C++ sense) by default, with
parameterization and instantiation directed by the compiler. We feel this
design is in line with a general trend towards automation in compiler
and language design.

%designing for type inference gives us two things
%- can design library with conscious tradeoffs
%- avoid premature optimization

\section{Language Design}


Static typing appears to have many advantages from a theoretical
standpoint: earlier error detection, generally better performance, and
support for more accurate tools are often cited in this context.
Nevertheless, developers are ``voting with their code'' for languages that
lack strong static typing disciplines. This is the phenomenon that Julia's
design addresses, and as such we must present our view of the advantages of
dynamic languages. In particular, we do not assume that \emph{every}
feature of these languages is equally important.
We hypothesize that the following forms of ``dynamism'' are
the most useful:

\begin{itemize}
\item The ability to run code at load time and compile time, eliminating
some of the distractions of build systems and configuration files.
\item A universal {\tt Any} type as the only true static type,
allowing the issue of static types to be ignored when desired.
\item Never rejecting code that is syntactically well-formed.
\item Behavior that depends only on run-time types (i.e. no static overloading).
\end{itemize}

We explicitly forgo the following features in the interest of preserving
the possibility of static typing in a reasonably broad category of
situations:

\begin{itemize}
\item Types themselves are immutable.
\item The type of a value cannot change over its lifetime.
\item Local variable environments are not reified.
\item Program code is immutable, but new code may be generated and executed at any time.
\item Not all bindings are mutable ({\tt const} identifiers are allowed).
\end{itemize}

This set of restrictions allows the compiler to see all uses of local
variables, and perform dataflow analysis on local variables using only
local information. This is important, since it allows user code to call
statically-unknown functions without interfering with optimizations done
around such call sites. Statically-unknown function calls arise in
many contexts, such as calling a function taken from an untyped data structure,
or dynamically dispatching a method call due to unknown argument types.

The core Julia language contains the following components:

\begin{enumerate}
\item A syntax layer, to translate surface syntax to a suitable
intermediate representation (IR).
\item A symbolic language and corresponding data structures for representing
certain kinds of types, and implementations of lattice operators ($meet$,
$join$, and $\leq$) for those types.
\item An implementation of generic functions and dynamic multiple dispatch
based on those types.
\item Compiler intrinsic functions for accessing the object model
(type definition, method definition, object allocation, element access,
testing object identity, and accessing type tags).
\item Compiler intrinsic functions for native arithmetic, bit string operations,
and calling native (C or Fortran) functions.
\item A mechanism for binding top-level names.
\end{enumerate}

The IR describes a function body as a sequence of assignment operations,
function calls, labels, and conditional branches. Julia's semantics
are those of a standard imperative language: statements are executed in order,
with function arguments evaluated eagerly. All values are conceptually
references, and are passed by reference as in LISP.

%% Julia's core evaluation semantics are particularly bland, because all of the
%% interesting work has been moved to the generic function system. Every
%% function definition is actually a definition of a method for some generic
%% function for some combination of argument types. The ``feel'' of the language
%% derives mostly from the fact that every function call is dynamically
%% dispatched to the most specific matching method definition, based on the
%% types of all arguments.


\subsection{Types}

Julia uses dynamic typing, which means that the universal type {\tt Any}
is the only static type. Our design philosophy is that types should be
quite expressive, but nearly invisible to the user. Julia
programmers must be able to ignore the type system if they do
not wish to make explicit use of it.

Julia treats types as symbolic descriptions of sets of values. Every value has
a unique, immutable, run-time implementation type. Objects carry type tags, and
types themselves are Julia objects that can be created and inspected
at run time.
Julia has five kinds of types:

\begin{enumerate}
\item abstract types, which may have explicitly-declared subtypes and supertypes
\item composite types (similar to C structs), which have named fields and
      explicitly-declared supertypes
\item bits types, whose values are represented as bit strings, and which may
      have explicitly-declared supertypes
\item tuples, immutable ordered collections of values. The type of a tuple is
defined recursively as a tuple of the types of the elements. Tuple types are
covariant in their element types. A tuple is used to represent the type of a
method's arguments.
\item union types, abstract types constructed from other types via set union
\end{enumerate}

Abstract types, composite types, and bits types may have parameters, which
makes it possible to express variants of a given type (for example, array types
with different element types). These types are all invariant with respect to
their parameters (i.e. two versions of the same type with different parameters
are simply different, and have no subtype or supertype relationship). Type
constructors are applied using curly braces, as in {\tt Array\{Float64,1\}}
(the {\tt Array} type is parameterized by element type and rank).

Bits types allow users to add new fixed-width number-like types and obtain the
same performance that primitive numeric types enjoy in other systems. Julia's
``built in'' numeric types are defined as bits types. Julia method dispatch
is based on types rather than field lookup, so whether a value is of a bits
type or composite type is a representation detail that is generally
invisible.

Tuple types may end in a special {\tt ...} type that
indicates any number of elements may be added. This is used to express the
types of variadic methods. For example the type {\tt (String, Int...)}
indicates a tuple where the first element is a String and any number of
trailing integers may be present.

Union types are used primarily to construct tight least upper bounds
when the inference algorithm needs to join unrelated types. For example,
a method might return an {\tt Int} or a {\tt String} in separate
arms of a conditional. In this case its type can be inferred as
{\tt Union(Int,String)}. Union types are also useful for defining
ad-hoc type hierarchies different from those imagined when the types
involved were first defined. For example, we could use
{\tt Union(Int,Range{Int})} as the type of array indexes, even though
there is no {\tt Index} type inherited by both constituents. Lastly,
union types can be used to declare methods applicable to multiple types.

%% The key to the utility of Julia's type system is its implementation of two
%% important functions: the subtype predicate, which determines whether one type
%% is a subset of another, and type intersection, which computes a type that is a
%% subtype of two given types. These functions form the basis of method dispatch
%% logic and type inference.

\subsection{Notational Conveniences}

An important goal is for users to be able to write Julia programs with
virtually no knowledge of type system details. Therefore we allow writing
parametric types without parameters, or omitting trailing parameters.
{\tt Array} refers to any kind of dense array, {\tt Array\{Float64\}} refers
to a Float64 Array of any rank, and {\tt Array\{Float64,2\}} refers to a
2-dimensional Float64 Array.

This design also makes it easy to add parameters to types later; existing
code does not need to be modified.

%%% NOTE: nominated for deletion by JWB
%% \subsection{Standard Type Hierarchy}
%% Here we present an excerpt from the standard library, showing how a few
%% important types are defined. The fields of composite types are redacted for
%% the sake of brevity. The {\tt <:} syntax indicates a declared subtype
%% relation.

%% \begin{verbatim}
%% abstract Type{T}
%% type AbstractKind  <: Type; end
%% type BitsKind      <: Type; end
%% type CompositeKind <: Type; end
%% type UnionKind     <: Type; end

%% abstract Number
%% abstract Real     <: Number
%% abstract Float    <: Real
%% abstract Integer  <: Real
%% abstract Signed   <: Integer
%% abstract Unsigned <: Integer

%% bitstype 32 Float32 <: Float
%% bitstype 64 Float64 <: Float

%% bitstype 8  Bool <: Integer
%% bitstype 32 Char <: Integer

%% bitstype 8  Int8   <: Signed
%% bitstype 8  Uint8  <: Unsigned
%% bitstype 16 Int16  <: Signed
%% bitstype 16 Uint16 <: Unsigned
%% bitstype 32 Int32  <: Signed
%% bitstype 32 Uint32 <: Unsigned
%% bitstype 64 Int64  <: Signed
%% bitstype 64 Uint64 <: Unsigned

%% abstract AbstractArray{T,N}
%% type Array{T,N} <: AbstractArray{T,N}; end
%% \end{verbatim}

% pattern emerges where ``user code'' has no types, and the more types
% written in code the more ``library-like'' it becomes (cite scala?)


\subsection{Generic Functions}

% TODO:
% talk about how this is a sneaky way to collect type information. since
% signatures specify dispatch behavior they are not redundant information.

The vast majority of Julia functions (in both the library and user programs)
are generic functions, meaning they contain multiple definitions or methods for
various combinations of argument types. When a generic function is applied,
the most specific definition that matches the run-time argument types is
invoked. Generic functions have appeared in several object systems in the past,
notably CLOS \cite{closoverview} and Dylan \cite{dylanlang}.
Julia is distinguished from these in that
it uses generic functions as its primary abstraction mechanism, putting it in
the company of research languages like Diesel \cite{dieselspec}. Aside
from being practical for mathematical styles of programming,
this design is satisfying also because it permits
expression of most of the popular patterns of object-oriented programming,
while leaving the core language with fewer distinct features.

\subsection{Method Definition}
Method definitions have a long (multi-line) form and a short form.

\begin{verbatim}
function iszero(x::Number)
    return x==0
end

iszero(x) = (x==0)
\end{verbatim}

A type declaration with {\tt ::} on an argument is a dispatch specification.
When types are omitted, the default is {\tt Any}.
A {\tt ::} expression may be added to any program expression, in which case
it acts as a run-time type assertion. As a special case, when {\tt ::} is
applied to a variable name in statement position (a construct which otherwise
has no effect) it means the variable \emph{always} has the specified type,
and values will be converted to that type (by calling {\tt convert}) on
assignment to the variable.

Note that there is no distinct type context; types are computed by ordinary
expressions
evaluated at run time. For example, {\tt f(x)::Int} is lowered to the
function call {\tt typeassert(f(x),Int)}.

Anonymous functions are written using the syntax {\tt x->x+1}.

Local variables are introduced implicitly by assignment. Modifying a
global variable requires a {\tt global} declaration.

Operators are simply functions with special calling syntax. Their
definitions look the same as those of ordinary functions, for example
{\tt +(x,y)~=~...}, or {\tt function~+(x,y)}.

When the last argument in a method signature is followed by {\tt ...}
the method accepts any number of arguments, and the last argument name
is bound to a tuple containing the tail of the argument list. The syntax
{\tt f(t...)} ``splices'' the contents of an iterable object {\tt t} as the
arguments to {\tt f}.

Generic functions are a natural fit for mathematical programming. For example,
consider implementing exponentiation (the {\tt \^{}} operator in Julia).
This function
lends itself to multiple definitions, specializing on both arguments
separately: there might be one definition for two floating-point numbers that
calls a standard math library routine, one definition for the case where the
second argument is an integer, and separate definitions for the case where the
first argument is a matrix. In Julia these signatures would be written as
follows:

\begin{verbatim}
function ^(x::Float64, p::Float64)
function ^(x, p::Int)
function ^(x::Matrix, p)
\end{verbatim}

\subsection{Parametric Methods}

It is often useful to refer to parameters of argument types inside methods,
and to specify constraints on those parameters for dispatch purposes.
Method parameters address these needs. These parameters behave a bit like
arguments, but they are always derived automatically from
the argument types and not specified explicitly by the caller.
The following signature presents a typical example:

\begin{verbatim}
function assign{T<:Integer}(a::Array{T,1}, i, n::T)
\end{verbatim}

This signature is applicable to 1-dimensional arrays whose element type is
some kind of integer, any type of second argument, and a third argument
that is the same type as the array's element type. Inside the method,
{\tt T} will be bound to the array element type.

The primary use of this construct is to write methods applicable to a
family of parametric types
(e.g. all integer arrays, or all numeric arrays)
despite invariance. The other use is
writing ``diagonal'' constraints as in the example above. Such diagonal
constraints significantly complicate the type lattice operators.


%%% NOTE: nominated for deletion by JWB
%% \subsection{Type Definition}

%% \begin{verbatim}
%% # abstract type
%% abstract Complex{T<:Real} <: Number

%% # composite type
%% type ComplexPair{T<:Real} <: Complex{T}
%%     re::T
%%     im::T
%% end

%% # bits type
%% bitstype 128 Complex128 <: Complex{Float64}

%% # type alias
%% typealias TwoOf{T} (T,T)
%% \end{verbatim}

\subsection{Constructors}

Composite types are applied as functions to construct instances.
The default constructor accepts values for each field as arguments.
Users may override the default constructor by writing method definitions
with the same name as the type inside the {\tt type} block. Inside the
{\tt type} block the identifier {\tt new} is bound to a pseudofunction
that actually constructs instances from field values. The constructor
for the {\tt Rational} type is a good example:

\begin{verbatim}
type Rational{T<:Integer} <: Real
    num::T
    den::T

    function Rational(num::T, den::T)
        if num == 0 && den == 0
            error("invalid rational: 0//0")
        end
        g = gcd(den, num)
        new(div(num, g), div(den, g))
    end
end
\end{verbatim}

This allows {\tt Rational} to enforce representation as a fraction in
lowest terms.

\subsection{Singleton Kinds}

A generic function's method table is effectively a dictionary where the keys
are types. This suggests that it should be just as easy to define or look up
methods with types themselves as with the types of values. Defining methods on
types directly is analogous to defining class methods in class-based object
systems. With multi-methods, definitions can be associated with combinations
of types, making it easy to represent properties not naturally owned by one
type.

To accomplish this, we introduce a special singleton kind {\tt Type\{T\}},
which contains the type {\tt T} as its only value.
%This is similar to the
%self-type pattern [cite], except along the type-of hierarchy rather than along
%the subtype-of hierarchy.
The result is a feature similar to {\tt eql}
specializers in CLOS, except only for types. An example use is defining
type traits:

\begin{verbatim}
typemax(::Type{Int64}) = 9223372036854775807
\end{verbatim}

This definition will be invoked by the call {\tt typemax(Int64)}. Note that
the name of a method argument can be omitted if it is not referenced.

Types are useful as method arguments in several other cases. One example is
file I/O, where a type can be used to specify what to read. The call
{\tt read(file,Int32)} reads a 4-byte integer and returns it as an {\tt Int32}
(a fact that the type inference process is able to discover). We find this
more elegant and convenient than systems where enums or special constants must
be used for this purpose, or where the type information is implicit
(e.g. through return-type overloading).

% TODO: note that this, together with appropriate specialization logic,
% gives the benefits of static parameters without bothering the user.

\subsection{Method Sorting and Ambiguity}
Methods are stored sorted by specificity, so the first matching method
(as determined by the subtype predicate) is always the correct one to invoke.
This means much of the dispatch logic is contained in the sorting process.
Comparing method signatures for specificity is not trivial. As one might
expect, the ``more specific''\footnote{Actually, ``not less specific'',
since specificity is a partial order.}
predicate is quite similar to the subtype
predicate, since a type that is a subtype of another is indeed more specific
than it. However, a few additional rules are necessary to capture the
intuitive concept of ``more specific''. In fact until this point
``more specific'' has had no formal meaning; its formal definition is
summarized as the disjunction of the following rules ($A$ is more specific
than $B$ if):

\begin{enumerate}
\item $A$ is a subtype of $B$
\item $A$ is of the form {\tt T\{P\}} and $B$ is of the form {\tt S\{Q\}}, and
$T$ is a subtype of $S$ for some parameter values
\item The intersection of $A$ and $B$ is nonempty, more specific than $B$, and
not equal to $B$, and $B$ is not more specific than $A$
\item $A$ and $B$ are tuple types, $A$ ends in a vararg ({\tt ...}) type,
and $A$ would be more specific than $B$ if its vararg type were expanded to
give it the same number of elements as $B$
\end{enumerate}

Rule 2 means that declared subtypes are always more specific than their
declared supertypes regardless of type parameters. Rule 3 is mostly useful for
union types: if $A$ is {\tt Union(Int32,String)} and $B$ is {\tt Number}, $A$
should
be more specific than $B$ because their intersection ({\tt Int32}) is clearly
more specific than $B$. Rule 4 means that argument types are more important for
specificity than argument count; if $A$ is {\tt (Int32...)} and $B$ is
{\tt (Number, Number)} then $A$ is more specific.

Julia uses \emph{symmetric} multiple dispatch, which means all argument types
are equally important. Therefore, ambiguous signatures are possible.
For example, given {\tt foo(x::Number, y::Int)} and
{\tt foo(x::Int, y::Number)} it is not clear which method to call when both
arguments are integers. We detect ambiguities when a method is added, by
looking for a pair of signatures with a non-empty intersection where neither
one is more specific than the other. A warning message is displayed for each
ambiguity, showing the user the computed type intersection so it is clear what
definition is missing. For example:

\begin{verbatim}
Warning: New definition foo(Int,Number) is
   ambiguous with foo(Number,Int). Make sure
   foo(Int,Int) is defined first.
\end{verbatim}


%%% NOTE: nominated for deletion by JWB
%% \subsection{Intrinsic Functions}

%% The run-time system contains a small number of primitive functions for tasks
%% like determining the type of a value, accessing fields of composite types, and
%% constructing values of each of the supported kinds of concrete types. There are
%% also arithmetic intrinsics corresponding to machine-level operations like
%% fixed-width integer addition, bit shifts, etc. These intrinsic functions are
%% implemented only in the code generator and do not have callable entry points.
%% They operate on bit strings, which are not first class values but can be
%% converted to and from Julia bits types via boxing and unboxing operations.

%% In our implementation, the core system also provides functions for constructing
%% arrays and accessing array elements. Although this is not strictly necessary,
%% we did not want to expose unsafe memory operations (e.g. load and store
%% primitives) in the language. In an earlier implementation, the core system
%% provided a bounds-checked {\tt Buffer} abstraction, but having both this type
%% and the user-level {\tt Array} type proved inconvenient and confusing.

\subsection{Iteration}

A {\tt for} loop is translated to a while loop with method calls according
to an iteration interface ({\tt start}, {\tt done}, and {\tt next}).

\begin{verbatim}
for i in range
    # body
end
\end{verbatim}

Becomes:

\begin{verbatim}
state = start(range)
while !done(range, state)
  (i, state) = next(range, state)
  # body
end
\end{verbatim}

This design for iteration was chosen because it is not tied to mutable
heap-allocated state, such as an iterator object that updates itself.

\subsection{Special Operators}

Special syntax is provided for certain functions.

\begin{tabular}{|l|l|}\hline
surface syntax     & lowered form \\\hline \hline
{\tt a[i, j]}      & {\tt ref(a, i, j)} \\\hline
{\tt a[i, j] = x}  & {\tt assign(a, x, i, j)} \\\hline
{\tt [a; b]}       & {\tt vcat(a, b)} \\\hline
{\tt [a, b]}       & {\tt vcat(a, b)} \\\hline
{\tt [a b]}        & {\tt hcat(a, b)} \\\hline
{\tt [a b; c d]}   & {\tt hvcat((2,2), a, b, c, d)}\\\hline
\end{tabular}


\subsection{Design Limitations}

In our design, type information always flows along with values, in the
forward control flow direction. This prevents us from doing certain tricks
that static type systems are capable of, such as return-type overloading.
Return-type overloading requires a robust notion of the type of a value
\emph{context}---the type expected or required of some term---in order to
select code on that basis. There are other cases where ``backwards'' type
flow might be desirable, such as determining the type of a container based
on the type of a value stored into it at a later program point. It may be
possible to get around this limitation in the future using inversion of
control---passing a function argument whose result type has already been
inferred, and using that type to construct a container before elements are
computed.

Modularity is a perennial difficulty with multiple dispatch, as any
function may apply to any type, and there is no point where functions or
types are closed to future definitions. Thus at the moment Julia is
essentially a whole-program compiler. We plan to implement a module system
that will at least allow code to control which name bindings and definitions
it sees. Such modules could be separately compiled to the extent that
programmers are willing to ask for their definitions to be ``closed''.

Lastly, at this time Julia uses a bit more memory than we would prefer.
Our compiler data structures, type information, and generated native code
take up more space than the compact bytecode representations used by many
dynamic languages.

\subsection{Extra Features}

\begin{enumerate}
\item symmetric coroutines
\item macros
\item ccall
\end{enumerate}

\section{Implementation}

Much of the implementation is organized around method dispatch. The dispatch
logic is both a large portion of the behavior of Julia functions, and the
entry point of the compiler's type inference and specialization logic.

\subsection{Method Caching and Specialization}

The first step of method dispatch is to look for the argument types in a
per-function cache. The cache has an entry for (almost) every set of concrete
types to which the function has been applied. Concrete types are hash-consed,
so they can be compared by simple pointer comparison. This makes cache lookup
faster than the $subtype$ predicate. As part of hash-consing, concrete types
are assigned small integer IDs. The ID of the first argument is used as a
primary key into a method cache, so when signatures differ only in the
type of the first argument a simple indexed lookup suffices.

On a cache miss, a slower search for the matching definition is performed using
$subtype$.
Then, type inference is invoked on the matching method using the types
of the actual arguments. The resulting type-annotated and optimized method is
stored in the cache. In this way, method dispatch is the primary source of type
information for the compiler.

\subsection{Method Specialization Heuristics}

Our aggressive use of code specialization has the obvious pitfall that it might
lead to excessive code generation, consuming memory and compile time. We found
that a few mild heuristics suffice to give a usable system with reasonable
resource requirements.

The first order of business is to ensure that the dispatch and specialization
process converges. The reason it might not is that our type inference algorithm
is implemented in Julia itself. Calling a method on a certain type $A$ can cause
the type inference code to call the same method on type $B$, where types
$A$ and $B$
follow an infinite ascending chain in either of two partial orders (the
$typeof$ order or the $subtype$ order). Singleton kinds are the most
prominent example, as type inference might attempt to successively consider
{\tt Int32}, {\tt Type\{Int32\}}, {\tt Type\{Type\{Int32\}\}}, and so on. We
stop this process by replacing any nestings of {\tt Type} with the
unspecialized version of {\tt Type} during method specialization (unless the
original method declaration actually specified a type like
{\tt Type\{Type\{Int32\}\}}).

The next heuristic avoids specializing methods for tuple types of every length.
Tuple types are cached as the intersection of the declared type of the method
slot with the generic tuple type {\tt (Any...)}. This makes the resulting cache
entry valid for any tuple argument, again unless the method declaration
contained a more specific tuple type. Note that all of these heuristics require
corresponding changes in the method cache lookup procedure, since they yield
cache entries that do not have to exactly match candidate arguments.

A similar heuristic is applied to variadic methods, where we wish to avoid
caching argument lists of every length. This is done by capping argument lists
at the length of the longest signature of any method in the same generic
function. The ``capping'' involves replacing the last argument with a
{\tt ...} type. Ideally, we want to form the biggest type that's not a
supertype of any other method signatures. However, this is not always possible
and the capped type might conflict with another signature. To deal with this
case, we find all non-empty intersections of the capped type with other
signatures, and add dummy cache entries for them. Hitting one of these entries
alerts the system that the arguments under consideration are not really in the
cache. Without the dummy entries, some arguments might incorrectly match the
capped type, causing the wrong method to be invoked.

The next heuristic concerns singleton kinds again. Because of the singleton
kind feature, every distinct type object ({\tt Any}, {\tt Number}, {\tt Int},
etc.) passed to a method might trigger a new specialization. However, most
methods are not ``class methods'' and are not concerned with type objects.
Therefore, if no method definition in a certain function involves {\tt Type}
for a certain argument slot, then that slot is not specialized for different
type objects.

Finally, we introduce a special type {\tt ANY} that can be used in a method
signature to hint that a slot should not be specialized. This is used in the
standard library in a small handful of places, and in practice is less
important than the heuristics described above.


\subsection{Type Inference}

Types of program expressions and variables are inferred by forward
dataflow analysis\footnote{Adding a reverse dataflow pass could potentially
improve type information, but we have not yet done this.}.
A key feature of this form of type inference is that variable types are
inferred at each use, since assignment is allowed to change the type of
a variable.
We determine a maximum fixed-point (MFP) solution using
Algorithm~\ref{alg1}, based on
Mohnen's graph-free dataflow analysis framework \cite{graphfree}. The basic
idea is to keep track of the state (the types of all variables) at each program
point, determine the effect of each statement on the state, and ensure that
type information from each statement eventually propagates to all other
statements reachable by control flow. We augment the basic algorithm with
support for mutually-recursive functions
(functions are treated as program points that might need to be revisited).

The origin of the type information used by the MFP algorithm is
evaluation of known functions over the type domain \cite{abstractinterp}.
This is done by the $eval$ subroutine. The $interpret$ subroutine calls
$eval$, and also handles assignment statements by returning the new types
of affected variables. Each known function
call is either to one of the small number of built-in functions, in which
case the result type is computed by a (usually trivial) hand-written
type transfer function, or to a generic function, in which case the result
type is computed by recursively invoking type inference. In the generic
function case, the inferred argument types are met ($\sqcap$) with the
signatures of each method definition. Matching methods are those where the
meet (greatest lower bound)
is not equal to the bottom type ({\tt None} in Julia).
Type inference is invoked on each matching
method, and the results are joined ($\sqcup$) together. The following equation
summarizes this process:

\[
T(f,t_{arg}) = \bigsqcup_{(s,g) \in f}T(g,t_{arg} \sqcap s)
\]

\noindent
$T$ is the type inference function.
$t_{arg}$ is the inferred argument tuple type. The tuples $(s,g)$
represent the signatures $s$ and their associated definitions $g$ within
generic function $f$.

Two optimizations are helpful here. First, it is rarely
necessary to consider all method definitions. Since methods are stored in
sorted order, as soon as the union of the signatures considered so far is a
supertype of $t_{arg}$, no more definitions need to be considered.
Second, the join operator employs \emph{widening} \cite{widening}:
if a type becomes too large it may simply return {\tt Any}. In this case
the recursive inference process may stop immediately.

% \subsubsection{MFP Dataflow Algorithm}

% \renewcommand{\algorithmicrequire}{\textbf{Input:}}
% \renewcommand{\algorithmicensure}{\textbf{Output:}}

\begin{algorithm}
\caption{Infer function return type}
\label{alg1}
\begin{algorithmic}
\REQUIRE function $F$, argument type tuple $A$, abstract execution stack $S$
\ENSURE result type $S.R$
\STATE $V \leftarrow$ set of all locally-bound names
\STATE $V_{a} \leftarrow$ argument names
\STATE $n \leftarrow length(F)$
\STATE $W \leftarrow \{1\}$ \COMMENT {set of program counters}
\STATE $P_r \leftarrow \emptyset$ \COMMENT {statements that recur}
\STATE $\forall v \in V, \Gamma[1][v] \leftarrow \text{Undef}$
\STATE $\forall i, \Gamma[1][V_{a}[i]] \leftarrow A[i]$  \COMMENT {type environment for statement 1}
\WHILE{$W \neq \emptyset$}
 \STATE $p \leftarrow \operatorname{choose}(W)$
 \REPEAT
  \STATE $W \leftarrow W - p$
  \STATE $new \leftarrow interpret(F[p],\Gamma[p],S)$
  \IF {$S.rec$}
   \STATE $P_r \leftarrow P_r \cup \{p\}$
   \STATE $S.rec \leftarrow \text{false}$
  \ENDIF
  \STATE $p^{\prime} \leftarrow p+1$
  \IF{$F[p] = $\texttt{(goto l)}}
   \STATE $p^{\prime} \leftarrow l$
  \ELSIF{$F[p] = $\texttt{(gotoif cond l)}}
   \IF {\NOT $new \leq \Gamma[l]$}
    \STATE $W \leftarrow W \cup \{l\}$
    \STATE $\Gamma[l] \leftarrow \Gamma[l] \sqcup new$
   \ENDIF
  \ELSIF{$F[p] = $\texttt{(return e)}}
   \STATE $p^{\prime} \leftarrow n+1$
   \STATE $r \leftarrow eval(e,\Gamma[p],S)$
   \IF {\NOT $r \leq S.R$}
    \STATE $S.R \leftarrow S.R \sqcup r$
    \STATE $W \leftarrow W \cup P_r$
   \ENDIF
  \ENDIF
  \IF{$p^{\prime} \leq n$ \AND \NOT $new \leq \Gamma[p^{\prime}]$}
   \STATE $\Gamma[p^{\prime}] \leftarrow \Gamma[p^{\prime}] \sqcup new$
   \STATE $p \leftarrow p^{\prime}$
  \ENDIF
 \UNTIL{$p^{\prime} = n+1$}
\ENDWHILE
\STATE {$S.rec \leftarrow P_r \neq \emptyset$}
\end{algorithmic}
\end{algorithm}

\subsubsection{Interprocedural Type Inference}

Type inference is invoked through ``driver'' Algorithm~\ref{alg2}
which manages mutual recursion and memoization of inference results.
A stack of abstract activation records is maintained and used to detect
recursion. Each function has a property $incomplete(F,A)$ indicating that
it needs to be revisited when new information is discovered about the
result types of functions it calls. The $incomplete$ flags collectively
represent a set analogous to $W$ in Algorithm~\ref{alg1}.

The outer loop in Algorithm~\ref{alg2} looks for an existing activation
record for its input function and argument types. If one is found, it
marks all records from that point to the top of the stack, identifying
all functions involved in the call cycle. These marks
are discovered in Algorithm~\ref{alg1} when $interpret$ returns, and all
affected functions are considered $incomplete$. Algorithm~\ref{alg2}
continues to re-run inference on incomplete functions, updating the
inferred result type, until no recursion occurs or the result type
converges.

\begin{algorithm}
\caption{Interprocedural type inference}
\label{alg2}
\begin{algorithmic}
\REQUIRE function $F$, argument type tuple $A$, abstract execution stack $S$
\ENSURE returned result type
\STATE $R \leftarrow \bot$
\IF {$recall(F,A)$ exists}
 \STATE $R \leftarrow recall(F,A)$
 \IF {\NOT $incomplete(F,A)$}
  \RETURN $R$
 \ENDIF
\ENDIF
\STATE $f \leftarrow S$
\WHILE {\NOT $\operatorname{empty}(f)$}
 \IF {$f.F$ is $F$ \AND $f.A=A$}
  \STATE $r \leftarrow S$
  \WHILE {\NOT $r=tail(f)$}
   \STATE $r.rec \leftarrow \text{true}$
   \STATE $r \leftarrow tail(r)$
  \ENDWHILE
  \RETURN $f.R$
 \ENDIF
 \STATE $f \leftarrow tail(f)$
\ENDWHILE
\STATE $S^{\prime} \leftarrow extend(S, Frame(F,A,R,rec=\text{false}))$
\STATE invoke Algorithm~\ref{alg1} on $F,A,S^{\prime}$
\STATE $recall(F,A) \leftarrow S^{\prime}.R$
\STATE $incomplete(F,A) \leftarrow (S^{\prime}.rec \land \neg(R=S^{\prime}.R))$
\RETURN $S^{\prime}.R$
\end{algorithmic}
\end{algorithm}

Because this algorithm approximates run-time behavior, we are free to
change it without affecting the behavior of user programs --- except
that they might run faster. One valuable improvement is
to attempt full evaluation of branch conditions, and remove branches
with constant conditions.

% \subsubsection{Type Transfer Functions}

%todo ?

\subsection{Lattice Operations}

Our type lattice is complicated by the presence of type parameters, unions,
and diagonal type constraints in method signatures. Fortunately, for our
purposes only the $\leq$ ($subtype$) relation needs to be computed accurately,
as it bears final responsibility for whether a method is applicable to
given arguments. Type union and intersection, used to estimate
least upper bounds and greatest lower bounds, respectively, may both be
conservatively approximated. If their results are too coarse, the
worst that can happen is performing method dispatch or type checks
at run time, since the inference process will simply conclude that it does
not know precise types.

A complication arises from the fact that our abstract domain is
available in a first-class fashion to user programs. When a program
contains a type-valued expression, we want to know which type it will
evaluate to, but this is not possible in general. Therefore in addition
to the usual \emph{type imprecision} (not knowing the type of a value),
we must also model \emph{type uncertainty}, where a type itself is
known imprecisely. A common example is application of the {\tt typeof}
primitive to a value of imprecise type. What is the abstract result of
{\tt typeof(x::Number)}? We handle this with a special type kind that
represents a \emph{range} rather than a point within the type lattice.
These kinds are essentially the type variables used in bounded
polymorphism \cite{boundedquant}. In this example, the
transfer function for {\tt typeof} is allowed to return
{\tt Type\{T<:Number\}}, where {\tt T} is a new type variable.


\subsubsection{Subtype Predicate}

See Algorithm~\ref{alg3}. Note that extensional type equality can be
computed as $(A\leq~B\land~B\leq~A)$, and this is used for types in
invariant context (i.e. type parameters). The algorithm uses subroutines
$p(A)$ which gives the parameters of type $A$, and $super(A)$ which gives
the declared supertype of $A$.

\begin{algorithm}
\caption{Subtype}
\label{alg3}
\begin{algorithmic}
\REQUIRE types $A$ and $B$
\ENSURE $A \leq B$
\IF {$A$ is a tuple type}
 \IF {$B$ is not a tuple type}
  \RETURN false
 \ENDIF
 \FOR {$i=1$ \TO $length(A)$}
  \IF {$A[i]$ is $T...$}
   \IF {$last(B)$ exists and is not $S...$}
    \RETURN false
   \ENDIF
   \RETURN $subtype(T,B[j])), i \leq j \leq length(B)$
  \ELSIF {$i > length(B)$ \OR \NOT $subtype(A[i],B[i])$}
   \RETURN false
  \ELSIF {$B[i]$ is $T...$}
   \RETURN $subtype(A[j],T)), i < j \leq length(A)$
  \ENDIF
 \ENDFOR
\ELSIF {$A$ is a union type}
 \RETURN $\forall t \in A, subtype(t,B)$
\ELSIF {$B$ is a union type}
 \RETURN $\exists t \in B, subtype(A,t)$
\ENDIF
\WHILE {$A \neq \texttt{Any}$}
 \IF {$typename(A) = typename(B)$}
  \RETURN {$subtype(p(A),p(B)) \land subtype(p(B),p(A))$}
 \ENDIF
 \STATE $A \leftarrow super(A)$
\ENDWHILE
\IF {$A$ is of the form {\tt Type\{T\}}}
 \RETURN $subtype(typeof(p(A)[1]),B)$
\ELSIF {$B$ is of the form {\tt Type\{T\}}}
 \STATE $B \leftarrow p(B)[1]$
 \RETURN $subtype(A,B) \land subtype(B,A)$
\ENDIF
\RETURN $B = \texttt{Any}$
\end{algorithmic}
\end{algorithm}


\subsubsection{Type Union}

Since our type system explicitly supports unions, the union of $T$ and
$S$ can be computed simply by constructing the type {\tt Union(T,S)}.
An obvious simplification is performed: if one of $T$ or $S$ is a
subtype of the other, it can be removed from the union. Nested union
types are flattened, followed by pairwise simplification.

\subsubsection{Type Intersection}

This is the difficult one: given types $T$ and $S$, we must try to compute
the smallest type $R$ such that
$\forall s, s \in T \land s \in S \Rightarrow s \in R$.
The conservative solution is to give up on finding the smallest such type, and
return \emph{some} type with this property. Simply returning $T$ or $S$
suffices for correctness, but in practice this algorithm
makes the type inference process nearly useless. A slightly better
algorithm is to check whether one argument is a subtype of the other, and
return the smaller type. It is also possible to determine quickly, in
many cases, that two types are disjoint, and return $\bot$. With these
two enhancements we start to obtain some useful type information. However,
we need to do much better to take full advantage of the framework set up
so far.

Our algorithm has two phases. First, the structures of the two input types
are analyzed in a manner similar to $subtype$, except a constraint
environment is built, with entries $T\leq S$ for type variables $T$ in
covariant contexts (tuples) and entries $T=S$ for type variables $T$ in
invariant contexts (type parameters). In the second phase the constraints
are solved with an algorithm similar to that
used by traditional polymorphic type systems \cite{MLtypeinf}.

The code for handling tuples and union types is similar to that in
Algorithm~\ref{alg3}, so we focus instead on intersecting types in the
nominal hierarchy (Algorithm~\ref{alg4}). The base case occurs when
the input types are from the same family, i.e. have the same
$typename$. All we need to do is visit each parameter to collect any
needed constraints, and otherwise check that the parameters are equal.
When a parameter is a type variable, it is effectively covariant, and
must be intersected with the corresponding parameter of the other type
to form the final result.

\begin{algorithm}
\caption{Intersection of nominal types}
\label{alg4}
\begin{algorithmic}
\REQUIRE types $A$ and $B$, current constraint environment
\ENSURE return $T$ such that $A \sqcap B \leq T$, updated environment
\IF {$typename(A) = typename(B)$}
 \STATE $pa \leftarrow \operatorname{copy}(p(A))$
 \FOR {$i=1$ \TO $length(p(A))$ }
  \IF {$p(A)[i]$ is a typevar}
   \STATE {add $(p(A)[i]=p(B)[i])$ to constraints}
  \ELSIF {$p(B)[i]$ is a typevar}
   \STATE {add $(p(B)[i]=p(A)[i])$ to constraints}
  \ENDIF
  \STATE $pa[i] \leftarrow intersect(p(A)[i],p(B)[i])$
 \ENDFOR
 \RETURN {$typename(A)\{pa...\}$}
\ELSE
 \STATE $sup \leftarrow intersect(super(A),B)$
 \IF {$sup = \bot$}
  \STATE $sup \leftarrow intersect(A,super(B))$
  \IF {$sup = \bot$}
   \RETURN $\bot$
  \ELSE
   \STATE $sub \leftarrow B$
  \ENDIF
 \ELSE
  \STATE $sub \leftarrow A$
 \ENDIF
 \STATE $E \leftarrow conform(sup, super\_decl(sub))$
 \IF {$E$ contains parameters not in $formals(sub)$}
  \RETURN $\bot$
 \ENDIF
 \RETURN $intersect(sub, typename(sub)\{E...\})$
\ENDIF
%\FORALL {$(S=U) \in E$}
% \STATE {$V \leftarrow $ value of $S$ in $sub$}
% \STATE {replace $U$ with $intersect(U,V)$}
%\ENDFOR
%\RETURN {instantiate $sub$ with $E$}
\end{algorithmic}
\end{algorithm}

%%% NOTE: nominated for deletion by JWB
%% \begin{algorithm}
%% \caption{Solve type variable constraints}
%% \label{alg5}
%% \begin{algorithmic}
%% \REQUIRE environment $X$ of pairs $T\leq S$ and $T=S$
%% \ENSURE environment $Y$ of unique variable assignments $T=S$, or failure
%% \STATE $Y \leftarrow \emptyset$
%% \STATE replace $(T\leq S) \in X$ with $(T=S)$ when $S$ is concrete
%% \FORALL {$(T=S) \in X$}
%%  \IF {$(T=R) \in X$ \AND $S\neq R$}
%%   \RETURN failure
%%  \ENDIF
%% \ENDFOR
%% \FORALL {$(T\leq S) \in X$}
%%  \IF {$(T=U) \in X$}
%%   \IF {\NOT $find(X,U)\leq S$}
%%    \RETURN failure
%%   \ELSE
%%    \STATE $X \leftarrow X - (T\leq S)$
%%   \ENDIF
%%  \ELSIF {$(T\leq U) \in X$ and $U$ not a variable}
%%   \STATE replace $U$ with $U\sqcap^{*}S$
%%   \STATE $X \leftarrow X - (T\leq S)$
%%  \ENDIF
%% \ENDFOR
%% \FORALL {variables $T$}
%%  \IF {$(T=U) \in X$}
%%   \STATE $Y \leftarrow Y \cup \{T=U\}$
%%  \ELSE
%%   \STATE $S \leftarrow \sqcap^{*} find(X,U), \forall (T\leq U) \in X$
%%   \IF {$S = \bot$}
%%    \RETURN failure
%%   \ENDIF
%%   \STATE $Y \leftarrow Y \cup \{T=S\}$
%%  \ENDIF
%% \ENDFOR
%% \end{algorithmic}
%% \end{algorithm}

When the argument types are not from the same family, we recur up the
type hierarchy to see if any supertype of one of the arguments matches
the other. If so, the recursion gives us the intersected supertype $sup$,
and we face the problem of mapping it to the family of the original argument
type. To do this, we first call subroutine $conform$, which takes two types
with the same structure and returns an environment $E$ mapping any
type variables in one to their corresponding components in the other.
$super\_decl(t)$ returns the type template used by $t$ to instantiate its
supertype. If all goes well, this tells us what parameters $sub$ would
have to be instantiated with to have supertype $sup$. If, however, $E$
contains type variables not controlled by $sub$, then there is no way
a type like $sub$ could have the required supertype, and the overall answer
is $\bot$.
Finally, we apply the base case to intersect $sub$ with the type obtained
by instantiating its family with parameter values in $E$.

We use a simple algorithm to solve the type parameter constraints.
Constraints $T\leq S$ where $S$ is a concrete type are converted to
$T=S$ to help sharpen the result type.
If there are any conflicting constraints ($T=S$ and $T=U$ where $S\neq U$),
the type intersection is empty. If each type variable has exactly one
constraint $T=U$, we can substitute $find(X,U)$ for each occurrence
of $T$ in the computed type intersection, and we have a final answer.
$find$ works in the \emph{union-find} sense, following chains of equalities
until we hit a non-variable or an unconstrained variable. Unconstrained
type variables may be left in place.

The remaining case is type variables with multiple constraints. Finding
a satisfying assignment requires intersecting all the upper bounds for
a variable. It is here that we choose to throw in the towel and switch
to a coarser notion of intersection, denoted by $\sqcap^{*}$.
Intersection is effectively the inner loop of type inference, so in the
interest of getting a reasonable answer quickly we might pick
$X\sqcap^{*}Y=X$. A few simple heuristics might as well be added; for
example cases like two non-parameterized types where one is an immediate
subtype of the other can be supported easily.

In our implementation, type intersection handles most of the
complexity surrounding type variables and parametric methods.
It is used to test applicability of parametric methods; since all
run-time argument lists are of concrete type, intersecting their types
with method signatures behaves like $subtype$, except static parameters
are also properly matched. If intersection returns $\bot$ or does not find
values for all static parameters for a method, the method is not applicable.
Therefore in practice we do not really have the freedom to implement
$\sqcap$ and $\sqcap^{*}$ any way that obeys our correctness property.
They must be at least as accurate as $subtype$ in the case where one
argument is concrete.


\subsubsection{Widening Operators}

Lattices used in practical program analyses often fail to obey the finite
chain condition necessary for the MFP algorithm to converge (i.e. they
are not of finite height) and ours is no exception.

Widening is applied in two places: by the join operator, and on every
recursive invocation of type inference.  When a union type becomes too
large (as determined by a cutoff), it is replaced with {\tt
  Any}. Tuple types lend themselves to two infinite chains: one in
depth ({\tt (Any,)}, {\tt ((Any,),)}, {\tt (((Any,),),)}, etc.)  and
one in length ({\tt (Any...,)}, {\tt (Any,Any...,)}, {\tt
  (Any,Any,Any...,)}, etc.). These chains are capped at arbitrary
cutoffs each time the inference process needs to construct a tuple
type.


%\subsubsection{Method Specificity Comparison}


\subsection{Code Generation and Optimization}

After type inference is complete, we annotate each expression with its
inferred type. We then run two symbolic optimization passes.
If the inferred argument types in a method call
indicate that a single method matches, we are free to inline that method.
For methods that return multiple values, inlining often yields
expressions that construct tuples and immediately take them apart. The
next optimization pass identifies these cases and removes the tuple
allocations.

The next set of optimizations is applied during code generation.
Our code generator targets the LLVM compiler framework \cite{LLVM}.
First, we examine uses of variables and assign local variables specific
scalar types where possible (LLVM uses a typed code representation).
The {\tt box} operations used to tag bit strings with types are done
lazily; they add a compile-time tag that causes generation of the
appropriate allocation code only when the value in question hits a context
that requires it (for example, assignment to an untyped data structure,
or being passed to an unknown function).

The code generator recognizes calls to key built-in and intrinsic functions,
and replaces them with efficient in-line code where possible. For example,
the {\tt is} function on mutable arguments yields a pointer comparison, and
{\tt typeof} might
yield a constant pointer value if the type of its argument is known.
Calls known to match single methods generate code to call the correct
method directly, skipping the dispatch process.

Finally, we run several of the optimization passes provided by LLVM.
This gives us all of the standard scalar optimizations, such as
strength reduction, dead code elimination, jump threading, and constant
folding. When
we are able to generate well-typed but messy code, LLVM gets us the
rest of the way to competitive performance. We have found that care
is needed in benchmarking: if the value computed by a loop is not
used, in some cases LLVM has been able to delete the whole thing.

%compile-time method lookup
%ccall intrinsic

% todo ?
\subsection{Run Time System}

\begin{enumerate}
\item lazy allocation of tuple types
\item storing compiler data serialized
\end{enumerate}

\section{Example Use Cases}

%By design, it is possible to write many programs in Julia that are
%entirely unsurprising to users of current systems such as Python, R, and
%\Matlab. Therefore, in this chapter we focus instead on
%more novel uses enabled by Julia's types and dispatch mechanism.

% these examples focus on unique features of julia. much julia code is
% written without using this sort of functionality, and looks quite similar
% to code in current systems.

\subsection{Numeric Type Promotion}

Numeric types and arithmetic are fundamental to all programming, but deserve
extra attention in the case of scientific computing.
In traditional compiled languages such as C, the arithmetic operators are the
most polymorphic ``functions'', and hence cannot be written in the language
itself. Arithmetic must be defined in the compiler, including contentious
decisions such as how to handle operations with mixed argument types.

In Julia, multiple dispatch is used to define arithmetic and type
promotion behaviors at the library level rather than in the compiler.
As a result, the system smoothly incorporates new
operators and numeric types with minimal work.

Four key utility functions comprise the type promotion system.
For simplicity, we consider only two-argument forms of promotion
although multi-argument promotion is also defined and used.

\begin{enumerate}
\item {\tt convert(T, value)} converts its second argument to type {\tt T}
\item {\tt promote\_rule(T1,T2) } defines which of two types is greater in
the promotion partial order
\item {\tt promote\_type(T1,T2) } uses {\tt promote\_rule} to determine which
type should be used for values of types {\tt T1} and {\tt T2}
\item {\tt promote(v1, v2) } converts its arguments to an appropriate type
and returns the results
\end{enumerate}

{\tt promote} is implemented as follows:

\begin{verbatim}
function promote{T,S}(x::T, y::S)
    (convert(promote_type(T,S),x),
     convert(promote_type(T,S),y))
end
\end{verbatim}

{\tt promote\_type} simply tries {\tt promote\_rule} with its arguments in
both orders, to avoid the need for repeated definitions:

\begin{verbatim}
function promote_type{T,S}(::Type{T}, ::Type{S})
    if applicable(promote_rule, T, S)
        return promote_rule(T,S)
    elseif applicable(promote_rule, S, T)
        return promote_rule(S,T)
    else
        error("no promotion exists")
    end
end
\end{verbatim}

{\tt convert} and {\tt promote\_rule} are implemented for each type. Two
such definitions for the {\tt Complex128} type are:

\begin{verbatim}
promote_rule(::Type{Complex128},
             ::Type{Float64}) = Complex128
convert(::Type{Complex128}, x::Real) =
        complex128(x, 0)
\end{verbatim}

With these definitions in place, a function may gain generic promotion
behavior by adding the following kind of definition:

\begin{verbatim}
+(x::Number, y::Number) = +(promote(x,y)...)
\end{verbatim}

This means that, given two numeric arguments where no more specific
definition matches, promote the arguments and retry the operation
(the {\tt ...} ``splices'' the two values returned by {\tt promote} into
the argument list).
The standard library contains such definitions for all basic arithmetic
operators.
For this recursion to terminate, we require only that each {\tt Number}
type implement {\tt +} for two arguments of that type, e.g.

\begin{verbatim}
+(x::Int64, y::Int64) = ...
+(x::Float64, y::Float64) = ...
+(x::Complex128, y::Complex128) = ...
\end{verbatim}

Therefore, each new type requires only one definition of each operator,
and a handful of {\tt convert} and {\tt promote\_rule} definitions.
If $n$ is the number of types and $m$ is the number of operators, a new
type requires $O(n+m)$ rather than $O(n\cdot m)$ definitions.

The reader will notice that uses of this mechanism involve multiple method
calls, as well as potentially expensive features such as tuple allocation
and argument splicing. Without a sufficient optimizing compiler, this
implementation would be completely impractical. Fortunately, through
type analysis, inlining, elision of unnecessary tuples, and lowering of
the {\tt apply} operation implied by {\tt ...}, Julia's compiler is able
to eliminate all of the overhead in most cases, ultimately yielding a
sequence of machine instructions comparable to that emitted by a
traditional compiler.

The most troublesome function is {\tt promote\_type}. For good performance,
we must elide calls to it, but doing so may be incorrect since the function
might throw an error. By fortunate coincidence though, the logic in
{\tt promote\_type} exactly mirrors the analysis done by type inference: it
only throws an error if no matching methods exist for its calls to
{\tt promote\_rule}, in which case type inference concludes that the
function throws an error regardless of which branch is taken.
{\tt applicable} is a built-in function known to be free of effects.
Therefore, whenever a sharp result type for {\tt promote\_type} can be
inferred, it is also valid to remove the unused arms of the conditional.


\subsection{Code Generation and Staged Functions}

The presence of types and an inference pass creates a new, intermediate
translation stage which may be customized (macros essentially customize
syntax processing, and object systems customize run time behavior).
This is the stage at which types are known, and it exists in Julia via
the compiler's method specialization machinery. Specialization may occur
at run time during dispatch, or at compile time when inference is able
to determine argument types accurately.
Running custom code at this stage has two tremendous effects:
first, optimized code can be generated for special cases, and
second, the type inference system can effectively be extended to be able to
make new type deductions relevant to the user's application.

For example, we might want to write functions that apply to two
arrays of different dimensionality, where the result has the higher of the
two argument dimensionalities. One such function is a ``broadcasting''
binary elementwise operator, that performs computations such as adding a
column vector to every column of a matrix, or adding a plane to every slice
of a 3-dimensional dataset. We can determine the shape of the result
array with the following function:

\begin{verbatim}
function promote_shape(s1::Tuple, s2::Tuple)
    if length(s1) > length(s2)
        return s1
    else
        return s2
    end
end
\end{verbatim}

The type system can easily express the types of array shapes, for example
{\tt (Int,Int)} and {\tt (Int,Int,Int)}. However, inferring a sharp result
type for this simple function is still challenging. The inference algorithm
would have to possess a theory of the {\tt length} and {\tt >} functions,
which is not easily done given that all Julia functions may be redefined
and overloaded with arbitrary methods.

One solution might be to allow the user to write some kind of compiler
extension or declaration. This approach is not ideal, since it might
result in duplicated information, or require the user to know more than
they want to about the type system.

Instead, this function can be written as a \emph{staged function} (or
more accurately in our case, a \emph{staged method}). This is a function
that runs at an earlier translation ``stage'', i.e. compile time, and
instead of returning a result value returns code that will compute the
result value when executed \cite{staging}.
Here is the staged version of
{\tt promote\_shape}\footnote{The {\tt @} denotes a macro invocation. At
present, staged methods are implemented by a macro, but full integration
into the language is planned.}:

\begin{verbatim}
@staged function promote_shape(s1::Tuple, s2::Tuple)
    if length(s1) > length(s2)
        quote return s1 end
    else
        quote return s2 end
    end
end
\end{verbatim}

The signature of this definition behaves exactly like any other method
signature: the type annotations denote run-time types for which the
definition is applicable. However, the body of the method will be invoked
on the \emph{types} of the arguments rather than actual arguments, and the
result of the body will be used to generate a new, more specialized
definition. For example, given arguments of types
{\tt (Int,Int)} and {\tt (Int,Int,Int)} the generated definition would be:

\begin{verbatim}
function promote_shape(s1::(Int,Int),
                       s2::(Int,Int,Int))
    return s2
end
\end{verbatim}

Observe that the type of this function is trivial to infer.

The staged function body runs as normal user code, so whatever definition
of {\tt >} is visible will be used, and the compiler does not have to know
how it behaves. Critically, the staged version of the function looks
similar to the normal version, requiring only the insertion of {\tt quote}
to mark expressions deferred to the next stage.

In the case where a program is already statically-typeable, staged
functions preserve that property. The types of the arguments to the
staged function will be known at compile time, so the custom code
generator can be invoked at compile time. Then the compiler may inline
the result or emit a direct call to the generated code, as usual.

Or, if the user does not require static compilation, the custom code
generator can be invoked at run time. Its results are cached for each new
combination of argument types, so compilation pauses are infrequent.

Thus we have a language with the convenience of run-time-only semantics,
which can be compiled just-in-time or ahead-of-time\footnote{The ahead-of-time
compiler is not yet implemented.} with minimal performance differences,
including custom code generation without the need for run-time {\tt eval}.
Most importantly, functions with complex type behavior can be implemented
in libraries without losing performance. Of course, ordinary Julia
functions may also have complex type behavior, and it is up to the
library designer to decide which functions should be staged.


%%% NOTE: nominated for deletion by JWB
%% \subsection{Generic Programming}

%% Support for generic programming is one of Julia's strengths.
%% Code in dynamic languages is often thought of as generic by default, due
%% to the absence of type restrictions, but this has its limits. First, many
%% systems, such as Common LISP, support optional type declarations to
%% improve performance. However, when this feature is used code usually
%% becomes monomorphic as a result. Second, some cases of generic
%% programming require the ability to specify behaviors that \emph{vary}
%% based on types, for example initializing a variable with the right kind
%% of container, or with an appropriate value for different numeric types.

%% Julia has neither of these problems. The first is solved both by
%% automatic specialization (which usually eliminates the need for
%% performance-seeking declarations), and static parameters, which allow
%% declarations containing type variables. The second is solved by the
%% ability to define type traits. Multiple dispatch is also helpful, as it
%% provides many ways to extend functions in the future.

%% As an example, here is how {\tt max} can be written for any array:

%% \begin{verbatim}
%% function max{T<:Real}(A::AbstractArray{T})
%%     v = typemin(T)
%%     for x in A
%%         if x > v
%%             v = x
%%         end
%%     end
%%     return v
%% end
%% \end{verbatim}

%% At the same time, we could provide an alternate implementation that makes
%% even fewer assumptions:

%% \begin{verbatim}
%% function max(A)
%%     v = typemin(eltype(A))
%%     ...
%% end
%% \end{verbatim}

%% This allows any container to expose its element type by implementing
%% {\tt eltype}. Here we do not have to know how the element type is
%% determined. In fact, this version would work for arrays equally well
%% as the first implementation, but we skip the opportunity to specify
%% that it is only defined for arrays with {\tt Real} elements.
%% We might also choose to call {\tt typemin} only if the container is
%% empty, and otherwise initialize {\tt v} with the first element. Then
%% {\tt max} would work on containers that do not implement {\tt eltype},
%% as long as they are never empty.

%% Flexible ad-hoc polymorphism plays a significant role in Julia's overall
%% performance. In scientific computing especially, important special cases
%% exist among the myriad datatypes that might appear in a program. Our
%% dispatch model permits writing definitions for more specialized cases
%% than most object-oriented languages.

%% For example, we have {\tt Array}, the type of dense arrays, and
%% {\tt SubArray}, an abstract array that references a contiguous subsection of
%% another array. The BLAS and LAPACK libraries allow the caller to specify
%% dimension strides, and we would like to call them for any arrays they
%% can handle. To do this, we can define the following types:

%% \begin{verbatim}
%% typealias Matrix{T}  Array{T,2}
%% typealias StridedMatrix{T,A<:Array}
%%           Union(Matrix{T}, SubArray{T,2,A})
%% \end{verbatim}

%% Then we can write definitions such as
%% {\tt *(StridedMatrix\{T\},} {\tt StridedMatrix\{T\})} for appropriate element types
%% {\tt T}. This replaces the custom dispatch schemes often implemented in
%% array programming systems.

% todo more?

\section{Evaluation}

\subsection{Performance}

We have compared the execution speed of Julia code to that of six other
languages: C++, Python, \Matlab, Octave, R, and JavaScript.
Figure~\ref{mbr}
\footnote{
  These measurements were carried out on a MacBook Pro with a 2.53GHz
  Intel Core 2 Duo CPU and 8GB of 1066MHz DDR3 RAM. The following
  versions were used: Python 2.7.1, \Matlab R2011a,
  Octave 3.4, R 2.14.2, V8 3.6.6.11.  The C++ baseline was compiled by
  GCC 4.2.1, taking best timing from all optimization levels.
  Native implementations of array operations, matrix
  multiplication, sorting, are used where available.
}
shows timings for five scalar microbenchmarks, and two
simple array benchmarks. All numbers are ratios relative to
the time taken by C++. The first five tests do not
reflect typical application performance in each environment; their only
purpose is to compare the code generation and execution for basic language
constructs, such as manipulating scalar quantities and referencing individual
array elements.

\begin{figure*}
\caption{Microbenchmark results (times relative to C++)}
\label{mbr}
\begin{center}

\begin{tabular}{|l|r|r|r|r|r|r|}\hline
test & Julia & Python & \Matlab & Octave & R & JavaScript \\
\hline \hline
fib        & 1.97 & 31.47 & 1336.37  & 2383.80 & 225.23 & 1.55 \\
\hline
parse\_int & 1.44 & 16.50 &  815.19  & 6454.50 & 337.52 & 2.17 \\
\hline
quicksort  & 1.49 & 55.84 &  132.71  & 3127.50 & 713.77 & 4.11 \\
\hline
mandel     & 5.55 & 31.15 &   65.44  &  824.68 & 156.68 & 5.67 \\
\hline
pi\_sum    & 0.74 & 18.03 &    1.08  &  328.33 & 164.69 & 0.75 \\
\hline
rand\_mat\_stat & 3.37 & 39.34 & 11.64 & 54.54 &  22.07 & 8.12 \\
\hline
rand\_mat\_mul  & 1.00 &  1.18 &  0.70 &  1.65 &   8.64 & 41.79 \\
\hline
\end{tabular}

\vspace{0.5cm}

\caption{Task-level benchmark results (times in seconds)}
\begin{tabular}{|l|r|r|r|r|}\hline
test              & Python run 1 & Python fastest & Julia run 1 & Julia fastest\\
\hline \hline
list and dispatch & 3.60         & 3.12           & 0.43        & 0.19 \\
\hline
CSV parse         & 0.06         & 0.06           & 0.49        & 0.17 \\
\hline
\end{tabular}

\end{center}
\end{figure*}


We can see why the standard libraries of these environments are
developed in C and Fortran. \Matlab has a JIT compiler
that works quite well in some cases, but is inconsistent, and
performs especially poorly on user-level function calls. The V8
JavaScript JIT compiler's performance is impressive. Anomalously, both
Julia and JavaScript seem to beat C++ on pi\_sum, but we have not yet
discovered why this might be.

The rand\_mat\_stat code manipulates many 5-by-5 matrices. Here the
performance gaps close, but the arrays are not large enough for
library time to dominate, so Julia's ability to specialize call sites
wins the day (despite the fact that most of the array library functions
involved are written in Julia itself).

The rand\_mat\_mul code demonstrates a case where time spent in BLAS \cite{blas}
dominates. \Matlab gets its edge from using a
multi-threaded BLAS (threading is available in the BLAS Julia uses,
but it was disabled when these numbers were taken). R may not be using
a well-tuned BLAS in this install; more efficient configurations are
probably possible.
JavaScript as typically deployed is not able to call the native BLAS code,
but the V8 compiler's work is respectable here.

Figure~\ref{mbr}\footnote{
Linux kernel 3.2.8 PC with 3.2GHz Intel Core i5 CPU, 4GB of RAM.
Python 2.7.2.} compares Julia and Python on some
more realistic ``task level'' benchmarks. The first test defines two
data types (classes in Python), then forms (by appending) a heterogeneous
array containing one million instances of each type. Then a method is called
on each object in the array. On the first run, Julia incurs some compilation
overhead and is about 8x faster. On future runs it is up to 16x faster.
Although the method calls cannot be optimized, Julia is still able to
gain an advantage likely due to use of native arithmetic for looping.

The second test reads each line of a 100000-line, 7MB CSV file, and
identifies and separates the comma-delimited fields. Python uses mature
C libraries for these tasks, and so is 3x to 8x faster than Julia. All of
the Julia library code is written in Julia, but this of course is no help
to an end user who only cares about application performance.

Julia is not yet able to cache generated native code, and so incurs a
startup time of about two seconds to compile basic library functions.
For some applications this latency is a barrier to deployment, and we plan
to address it in the future.


\subsection{Effectiveness of Specialization Heuristics}

Given our implementation strategy, excessive compilation and corresponding
memory use are potential performance concerns. In Figure~\ref{ncomp}
we present the number of method compilations performed on startup, and
after running a test suite. From the second row of the table to the bottom,
each of three specialization heuristics is successively enabled to
determine its effect on compiler workload. In the last table row, each
method is compiled just once.

The heuristics are able to elide about 12\% of compilations. This is
not a large fraction, but it is satisfying given that the heuristics can
be computed easily, and only by manipulating types. On average, each
method is compiled about 2.5 times.

\begin{figure}
\caption{Number of methods compiled}
\label{ncomp}
\begin{center}
\begin{tabular}{|l|r|r|}\hline
    & at startup & after test suite \\
\hline \hline
no heuristics & 396 & 2245 \\
\hline
manual hints & 379 & 2160 \\
\hline
tuple widening & 357 & 1996 \\
\hline
vararg widening & 355 & 1970 \\
\hline
no specialization & 267 & 766 \\
\hline
\end{tabular}
\end{center}
\end{figure}

Memory usage is not unreasonable for modern machines: on a 64-bit platform
Julia uses about 50MB of memory on startup, and after loading several
libraries and working for a while memory use tends to level off around
150-200MB. Pointer-heavy data structures consume a lot of space on
64-bit platforms. To mitigate this problem, we store ASTs and type
information in a compact serialized format, and deserialize structures
when the compiler needs them.


\subsection{Effectiveness of Type Inference}

It is interesting to count compiled expressions for which
a concrete type can be inferred. In some sense, this tells us ``how close''
Julia is to being statically typed, though in our case this is a property
of both the language implementation and the standard library.
In a run of our test suite, code was generated for 135375 expressions.
Of those, 84127 (62\%) had a type more specific than {\tt Any}. Of those,
80874 (96\%) had a concrete static type.

This suggests that use of dynamic typing is fairly popular, even though
we try to avoid it to some extent in the standard library. Still, more
than half of our code is well-typed. The numbers also suggest that,
despite careful use of a rich lattice, typing tends to be an all-or-nothing
affair. But, it is difficult to estimate the effect of the 4\%
abstractly-typed expressions on the other 96\%, not to mention the potential
utility of abstract inferred types in code that was not actually
compiled.

These numbers are somewhat inaccurate, as they include dead code, and
it may be the case that better-typed methods tend to be recompiled either
more or less often, biasing the numbers.


\subsection{Productivity}

Our implementation of Julia consists of 11000 lines of C, 4000 lines
of C++, and 3500 lines of Scheme (here we are not counting code in
external libraries such as BLAS and LAPACK).  Thus we have
significantly less low-level code to maintain than most scripting
languages.  Our standard library is roughly 25000 lines of Julia code.
The standard library provides around 300 numerical functions of the
sort found in all technical computing environments. We suspect that
our library is one of the most compact implementations of this body of
functionality.

At this time, every contributor except the core developers is a ``new
user'' of Julia, having known of the language for no more than six
months.  Despite this, our function library has received several
significant community contributions, and numerous smaller ones. We
take this as encouraging evidence that Julia is productive and easy to
learn.

%~12000 lines extras
%~3000 lines examples
%~4000 lines of test code

\section{Parallelism}

%% VBS: What do we say about parallelism. Do we stay quiet, deflect it
%% to later, or talk about broad principles in our approach to parallelism?

\section{Standard library}

\section{Community}

Julia was publicly announced in February 2012. Our goals and work so
far seemed to strike a chord, as we have seen a significant community
start to grow in the short time since then.

Julia is an open source project, with all code hosted on {\tt github}.
It has attracted 550 mailing list subscribers, 1500 github followers,
180 forks, and more than 50 total contributors. Text editor support
has been implemented for emacs, vim, and textmate.
Github recognizes Julia as the language of source files ending in
{\tt .jl}, and can syntax highlight Julia code listings.
We are currently \#79 in github's language popularity ranking.

Several community projects are underway: two plotting packages,
interfaces to arbitrary-precision arithmetic library GMP,
bit arrays, linear programming, image processing, polynomials,
GPU code generation, a statistics library, and a web-based interactive
environment. A package management framework will soon be in place.

%So far we detect enthusiasm for the ability to use declarative
%information and metaprogramming without giving up convenience for
%mathematical applications.
We hope Julia is part of a new generation of dynamic languages that not
only run faster, but foster more cooperation between the programmer
and compiler, pushing the standard of productivity ever higher.

% we have much work to do


% -------------------------------------------------------------


%\appendix
%\section{Appendix Title}

%This is the text of the appendix, if you need one.

\acks
Department of Energy grant DE-SC0002099
National Science Foundation grant CCF-0832997

% We recommend abbrvnat bibliography style.

\bibliographystyle{abbrvnat}

% The bibliography should be embedded for final submission.
%\bibliography{julia}

\begin{thebibliography}{16}
\softraggedright

\providecommand{\natexlab}[1]{#1}
\providecommand{\url}[1]{\texttt{#1}}
\expandafter\ifx\csname urlstyle\endcsname\relax
  \providecommand{\doi}[1]{doi: #1}\else
  \providecommand{\doi}{doi: \begingroup \urlstyle{rm}\Url}\fi

\bibitem[Bolz et~al.(2009)Bolz, Cuni, Fijalkowski, and Rigo]{pypyjit}
C.~F. Bolz, A.~Cuni, M.~Fijalkowski, and A.~Rigo.
\newblock Tracing the meta-level: Pypy's tracing jit compiler.
\newblock In \emph{Proceedings of the 4th workshop on the Implementation,
  Compilation, Optimization of Object-Oriented Languages and Programming
  Systems}, ICOOOLPS '09, pages 18--25, New York, NY, USA, 2009. ACM.
\newblock ISBN 978-1-60558-541-3.
%\newblock \doi{http://doi.acm.org/10.1145/1565824.1565827}.
%\newblock URL \url{http://doi.acm.org/10.1145/1565824.1565827}.

\bibitem[Cardelli and Wegner(1985)]{boundedquant}
L.~Cardelli and P.~Wegner.
\newblock On understanding types, data abstraction, and polymorphism.
\newblock \emph{ACM Comput. Surv.}, 17\penalty0 (4):\penalty0 471--523, Dec.
  1985.
\newblock ISSN 0360-0300.
%\newblock \doi{10.1145/6041.6042}.
%\newblock URL \url{http://doi.acm.org/10.1145/6041.6042}.

\bibitem[Chambers(2005)]{dieselspec}
C.~Chambers.
\newblock The diesel language specification and rationale: Version 0.1.
\newblock February 2005.

\bibitem[Cousot and Cousot(1977)]{abstractinterp}
P.~Cousot and R.~Cousot.
\newblock Abstract interpretation: a unified lattice model for static analysis
  of programs by construction or approximation of fixpoints.
\newblock In \emph{Proceedings of the 4th ACM SIGACT-SIGPLAN symposium on
  Principles of programming languages}, POPL '77, pages 238--252, New York, NY,
  USA, 1977. ACM.
%\newblock \doi{10.1145/512950.512973}.
%\newblock URL \url{http://doi.acm.org/10.1145/512950.512973}.

\bibitem[Cousot and Cousot(1992)]{widening}
P.~Cousot and R.~Cousot.
\newblock Comparing the galois connection and widening/narrowing approaches to
  abstract interpretation.
\newblock In M.~Bruynooghe and M.~Wirsing, editors, \emph{Programming Language
  Implementation and Logic Programming}, volume 631 of \emph{Lecture Notes in
  Computer Science}, pages 269--295. Springer Berlin / Heidelberg, 1992.
\newblock ISBN 978-3-540-55844-6.

\bibitem[DeMichiel and Gabriel(1987)]{closoverview}
L.~DeMichiel and R.~Gabriel.
\newblock The common lisp object system: An overview.
\newblock In J.~Bézivin, J.-M. Hullot, P.~Cointe, and H.~Lieberman, editors,
  \emph{ECOOP’ 87 European Conference on Object-Oriented Programming}, volume
  276 of \emph{Lecture Notes in Computer Science}, pages 151--170. Springer
  Berlin / Heidelberg, 1987.
\newblock ISBN 978-3-540-18353-2.

\bibitem[Gomez(1999)]{scilab}
C.~Gomez, editor.
\newblock \emph{Engineering and Scientific Computing With Scilab}.
\newblock Birkh{\"a}user, 1999.

\bibitem[Ihaka and Gentleman(1996)]{Rlang}
R.~Ihaka and R.~Gentleman.
\newblock R: A language for data analysis and graphics.
\newblock \emph{Journal of Computational and Graphical Statistics}, 5:\penalty0
  299--314, 1996.
\newblock ISSN 1061-8600.

\bibitem[J{\o}rring and Scherlis(1986)]{staging}
U.~J{\o}rring and W.~L. Scherlis.
\newblock Compilers and staging transformations.
\newblock In \emph{Proceedings of the 13th ACM SIGACT-SIGPLAN symposium on
  Principles of programming languages}, POPL '86, pages 86--96, New York, NY,
  USA, 1986. ACM.
%\newblock \doi{10.1145/512644.512652}.
%\newblock URL \url{http://doi.acm.org/10.1145/512644.512652}.

\bibitem[Lattner and Adve(2004)]{LLVM}
C.~Lattner and V.~Adve.
\newblock {LLVM: A Compilation Framework for Lifelong Program Analysis \&
  Transformation}.
\newblock In \emph{{Proceedings of the 2004 International Symposium on Code
  Generation and Optimization (CGO'04)}}, Palo Alto, California, Mar 2004.

\bibitem[Lawson et~al.(1979)Lawson, Hanson, Kincaid, and Krogh]{blas}
C.~L. Lawson, R.~J. Hanson, D.~R. Kincaid, and F.~T. Krogh.
\newblock Basic linear algebra subprograms for Fortran usage.
\newblock \emph{ACM Trans. Math. Softw.}, 5\penalty0 (3):\penalty0 308--323,
  Sept. 1979.
\newblock ISSN 0098-3500.
%\newblock \doi{10.1145/355841.355847}.
%\newblock URL \url{http://doi.acm.org/10.1145/355841.355847}.

\bibitem[Mohnen(2002)]{graphfree}
M.~Mohnen.
\newblock A graph—free approach to data—flow analysis.
\newblock In R.~Horspool, editor, \emph{Compiler Construction}, volume 2304 of
  \emph{Lecture Notes in Computer Science}, pages 185--213. Springer Berlin /
  Heidelberg, 2002.
\newblock ISBN 978-3-540-43369-9.

\bibitem[Murphy(1997)]{Octave}
M.~Murphy.
\newblock Octave: A free, high-level language for mathematics.
\newblock \emph{Linux J.}, 1997, July 1997.
\newblock ISSN 1075-3583.
%\newblock URL \url{http://dl.acm.org/citation.cfm?id=326876.326884}.

\bibitem[Robin and Milner(1978)]{MLtypeinf}
Robin and Milner.
\newblock A theory of type polymorphism in programming.
\newblock \emph{Journal of Computer and System Sciences}, 17\penalty0
  (3):\penalty0 348 -- 375, 1978.
\newblock ISSN 0022-0000.
%\newblock \doi{10.1016/0022-0000(78)90014-4}.
%\newblock URL
%  \url{http://www.sciencedirect.com/science/article/pii/0022000078900144}.

\bibitem[Shalit(1996)]{dylanlang}
A.~Shalit.
\newblock \emph{The Dylan reference manual: the definitive guide to the new
  object-oriented dynamic language}.
\newblock Addison Wesley Longman Publishing Co., Inc., Redwood City, CA, USA,
  1996.
\newblock ISBN 0-201-44211-6.

\bibitem[van~der Walt et~al.(2011)van~der Walt, Colbert, and Varoquaux]{numpy}
S.~van~der Walt, S.~C. Colbert, and G.~Varoquaux.
\newblock The numpy array: a structure for efficient numerical computation.
\newblock \emph{CoRR}, abs/1102.1523, 2011.

\end{thebibliography}


\end{document}
