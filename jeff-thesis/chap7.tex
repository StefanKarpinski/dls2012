\chapter{Conclusion and Project Status}

Julia was publicly announced in February 2012. Our goals and work so
far seemed to strike a chord, as we have seen a significant community
start to grow in the short time since then.

Julia is an open source project, with all code hosted on github \cite{github}.
We have over 450 mailing list subscribers, 1420 github followers,
170 forks, and more than 50 total contributors. Text editor support
has been implemented for emacs, vim, and textmate.
Github recognizes Julia as the language of source files ending in
{\tt .jl}, and can syntax highlight Julia code listings.
We are currently \#80 in github's language popularity ranking, up
from \#89 a month ago.

Several community projects are underway: two plotting packages,
interfaces to arbitrary-precision arithmetic library GMP,
bit arrays, linear programming, image processing, polynomials,
GPU code generation, a statistics library, and a web-based interactive
environment.

%So far we detect enthusiasm for the ability to use declarative
%information and metaprogramming without giving up convenience for
%mathematical applications. 
We hope Julia is part of a new generation of dynamic languages that not
only run faster, but foster more cooperation between the programmer
and compiler, pushing the standard of productivity ever higher.

% we have much work to do
