\chapter{Language Design}

Static typing appears to have many advantages from an objective, theoretical
standpoint: earlier error detection, generally better performance, and
support for more accurate tools are often cited in this context.
Nevertheless, developers are ``voting with their code'' for languages that
lack strong static typing disciplines. This is the phenomenon that Julia's
design addresses, and as such we must present our view of the advantages of
dynamic languages. In particular, we do not assume that \emph{every}
feature of these languages is equally important.
We hypothesize that the following forms of ``dynamism'' are
the most useful:

\begin{itemize}
\item The ability to run code at load time and compile time, eliminating
some of the distractions of build systems and configuration files.
\item A universal {\tt Any} type as the only true static type,
allowing the issue of static types to be ignored when desired.
\item Never rejecting code that is syntactically well-formed.
\item Behavior that depends only on run-time types (unlike, for example, C++,
where virtual methods are dispatched by run-time type and function overloads
are dispatched by static type).
\end{itemize}

We explicitly forgo the following features in the interest of preserving
the possibility of static typing in a reasonably broad category of
situations:

\begin{itemize}
\item Types themselves are immutable.
\item The type of a value cannot change over its lifetime.
\item Local variable environments are not reified.
\item Program code is immutable, but new code may be generated and executed at any time.
\item Not all bindings are mutable ({\tt const} identifiers are allowed).
\end{itemize}

This set of restrictions allows the compiler to see all uses of local
variables, and perform dataflow analysis on local variables using only
local information. This is important, since it allows user code to call
statically-unknown functions without interfering with optimizations done
around such call sites. Statically-unknown function calls arise in
many contexts, such as calling a function taken from an untyped data structure,
or dynamically dispatching a method call due to unknown argument types.

The core Julia language contains the following components:

\begin{enumerate}
\item A syntax layer, to translate surface syntax to a suitable
intermediate representation (IR).
\item A symbolic language and corresponding data structures for representing
certain kinds of types, and implementations of lattice operators ($meet$,
$join$, and $\leq$) for those types.
\item An implementation of generic functions and dynamic multiple dispatch
based on those types.
\item Compiler intrinsic functions for accessing the object model
(type definition, method definition, object allocation, element access,
testing object identity, and accessing type tags).
\item Compiler intrinsic functions for native arithmetic, bit string operations,
and calling native (C or FORTRAN) functions.
\item A mechanism for binding top-level names.
\end{enumerate}

The IR describes a function body as a sequence of assignment operations,
function calls, labels, and conditional branches. Julia's semantics
are those of a standard imperative language: statements are executed in order,
with function arguments evaluated eagerly. All values are conceptually
references, and are passed by reference as in LISP.

Julia's core evaluation semantics are particularly bland, because all of the
interesting work has been moved to the generic function system. Every
function definition is actually a definition of a method for some generic
function for some combination of argument types. The ``feel'' of the language
derives mostly from the fact that every function call is dynamically
dispatched to the most specific matching method definition, based on the
types of all arguments.


\section{Types}

Julia uses dynamic typing, which means that the universal type {\tt Any}
is the only static type. Our design philosophy is that types should be
quite powerful and expressive, but nearly invisible to the user. Julia
programmers must be able to ignore the type system completely if they do
not wish to make explicit use of its functionality.

Julia treats types as symbolic descriptions of sets of values. Every value has
a unique, immutable, run-time implementation type. Objects carry type tags, and
types themselves are Julia objects that can be manipulated at run time.
Julia has five kinds of types:

\begin{enumerate}
\item abstract types, which may have explicitly-declared subtypes and supertypes
\item composite types (similar to C structs), which have named fields and
      explicitly-declared supertypes
\item bits types, whose values are represented as bit strings, and which may
      have explicitly-declared supertypes
\item tuples, immutable ordered collections of values. The type of a tuple is
defined recursively as a tuple of the types of the elements. Tuple types are
covariant in their element types. A tuple is used to represent the type of a
method's arguments.
\item union types, abstract types constructed from other types via set union
\end{enumerate}

Abstract types, composite types, and bits types may have parameters, which
makes it possible to express variants of a given type (for example, array types
with different element types). These types are all invariant with respect to
their parameters (i.e. two versions of the same type with different parameters
are simply different, and have no subtype or supertype relationship). Type
constructors are applied using curly braces, as in {\tt Array\{Float64,1\}}
(the {\tt Array} type is parameterized by element type and rank).

Bits types allow users to add new fixed-width number-like types and obtain the
same performance that primitive numeric types enjoy in other systems. Julia's
``built in'' numeric types are defined as bits types. Julia method dispatch
is based on types rather than field lookup, so whether a value is of a bits
type or composite type is a representation detail that is generally
invisible.

As an extra complexity, tuple types may end in a special {\tt ...} type that
indicates any number of elements may be added. This is used to express the
types of variadic methods. For example the type {\tt (String, Int...)}
indicates a tuple where the first element is a String and any number of
trailing integers may be present.

Union types are used primarily to construct tight least upper bounds
when the inference algorithm needs to join unrelated types. For example,
a method might return an {\tt Int} or a {\tt String} in separate
arms of a conditional. In this case its type can be inferred as
{\tt Union(Int,String)}. Union types are also useful for defining
ad-hoc type hierarchies different from those imagined when the types
involved were first defined. For example, we could use
{\tt Union(Int,Range{Int})} as the type of array indexes, even though
there is no {\tt Index} type inherited by both constituents. Lastly,
union types can be used to declare methods applicable to multiple types.

The key to the utility of Julia's type system is its implementation of two
important functions: the subtype predicate, which determines whether one type
is a subset of another, and type intersection, which computes a type that is a
subtype of two given types. These functions form the basis of method dispatch
logic and type inference.

\subsection{Notational Conveniences}

An important goal is for users to be able to write Julia programs with
virtually no knowledge of type system details. Therefore we allow writing
parametric types without parameters, or omitting trailing parameters.
{\tt Array} refers to any kind of dense array, {\tt Array\{Float64\}} refers
to a Float64 Array of any rank, and {\tt Array\{Float64,2\}} refers to a
2-dimensional Float64 Array.

This design also makes it easy to add parameters to types later; existing
code does not need to be modified.

\subsection{Standard Type Hierarchy}
Here we present an excerpt from the standard library, showing how a few
important types are defined. The fields of composite types are redacted for
the sake of brevity. The {\tt <:} syntax indicates a declared subtype
relation.

\vspace{-0.2in}
\begin{singlespace}
\begin{verbatim}
abstract Type{T}
type AbstractKind  <: Type; end
type BitsKind      <: Type; end
type CompositeKind <: Type; end
type UnionKind     <: Type; end

abstract Number
abstract Real     <: Number
abstract Float    <: Real
abstract Integer  <: Real
abstract Signed   <: Integer
abstract Unsigned <: Integer

bitstype 32 Float32 <: Float
bitstype 64 Float64 <: Float

bitstype 8  Bool <: Integer
bitstype 32 Char <: Integer

bitstype 8  Int8   <: Signed
bitstype 8  Uint8  <: Unsigned
bitstype 16 Int16  <: Signed
bitstype 16 Uint16 <: Unsigned
bitstype 32 Int32  <: Signed
bitstype 32 Uint32 <: Unsigned
bitstype 64 Int64  <: Signed
bitstype 64 Uint64 <: Unsigned

abstract AbstractArray{T,N}
type Array{T,N} <: AbstractArray{T,N}; end
\end{verbatim}
\end{singlespace}

% pattern emerges where ``user code'' has no types, and the more types
% written in code the more ``library-like'' it becomes (cite scala?)


\section{Syntax}

Julia has a simple block-structured syntax, with notation for type
and method definition, control flow, and special syntax for important
operators.

\subsection{Method Definition}
Method definitions have a long (multi-line) form and a short form.

\vspace{-0.2in}
\begin{singlespace}
\begin{verbatim}
function iszero(x::Number)
    return x==0
end

iszero(x) = (x==0)
\end{verbatim}
\end{singlespace}

A type declaration with {\tt ::} on an argument is a dispatch specification.
When types are omitted, the default is {\tt Any}.
A {\tt ::} expression may be added to any program expression, in which case
it acts as a run-time type assertion. As a special case, when {\tt ::} is
applied to a variable name in statement position (a construct which otherwise
has no effect) it means the variable \emph{always} has the specified type,
and values will be converted to that type (by calling {\tt convert}) on
assignment to the variable.

Note that there is no distinct type context; types are computed by ordinary
expressions
evaluated at run time. For example, {\tt f(x)::Int} is lowered to the
function call {\tt typeassert(f(x),Int)}.

Anonymous functions are written using the syntax {\tt x->x+1}.

Local variables are introduced implicitly by assignment. Modifying a
global variable requires a {\tt global} declaration.

Operators are simply functions with special calling syntax. Their
definitions look the same as those of ordinary functions, for example
{\tt +(x,y)~=~...}, or {\tt function~+(x,y)}.

When the last argument in a method signature is followed by {\tt ...}
the method accepts any number of arguments, and the last argument name
is bound to a tuple containing the tail of the argument list. The syntax
{\tt f(t...)} ``splices'' the contents of an iterable object {\tt t} as the
arguments to {\tt f}.

\subsubsection{Parametric Methods}

It is often useful to refer to parameters of argument types inside methods,
and to specify constraints on those parameters for dispatch purposes.
Method parameters address these needs. These parameters behave a bit like
arguments, but they are always derived automatically from
the argument types and not specified explicitly by the caller.
The following signature presents a typical example:

\begin{verbatim}
function assign{T<:Integer}(a::Array{T,1}, i, n::T)
\end{verbatim}

This signature is applicable to 1-dimensional arrays whose element type is
some kind of integer, any type of second argument, and a third argument
that is the same type as the array's element type. Inside the method,
{\tt T} will be bound to the array element type.

The primary use of this construct is to write methods applicable to a
family of parametric types
(e.g. all integer arrays, or all numeric arrays)
despite invariance. The other use is
writing ``diagonal'' constraints as in the example above. Such diagonal
constraints significantly complicate the type lattice operators.


\subsection{Control Flow}

\vspace{-0.2in}
\begin{singlespace}
\begin{verbatim}
if condition1 || (a && b)
    # single line comment
elseif !condition2
else
    # otherwise
end

while condition
    # body
end

for i in range
    # body
end
\end{verbatim}
\end{singlespace}

A {\tt for} loop is translated to a while loop with method calls according
to the iteration interface ({\tt start}, {\tt done}, and {\tt next}):

\vspace{-0.2in}
\begin{singlespace}
\begin{verbatim}
state = start(range)
while !done(range, state)
  (i, state) = next(range, state)
  # body
end
\end{verbatim}
\end{singlespace}

This design for iteration was chosen because it is not tied to mutable
heap-allocated state, such as an iterator object that updates itself.

\subsection{Special Operators}

Special syntax is provided for certain functions.

\begin{singlespace}
\begin{tabular}{|l|l|}\hline
surface syntax     & lowered form \\\hline \hline
{\tt a[i, j]}      & {\tt ref(a, i, j)} \\\hline
{\tt a[i, j] = x}  & {\tt assign(a, x, i, j)} \\\hline
{\tt [a; b]}       & {\tt vcat(a, b)} \\\hline
{\tt [a, b]}       & {\tt vcat(a, b)} \\\hline
{\tt [a b]}        & {\tt hcat(a, b)} \\\hline
{\tt [a b; c d]}   & {\tt hvcat((2,2), a, b, c, d)}\\\hline
\end{tabular}
\end{singlespace}

\subsection{Type Definition}

\vspace{-0.2in}
\begin{singlespace}
\begin{verbatim}
# abstract type
abstract Complex{T<:Real} <: Number

# composite type
type ComplexPair{T<:Real} <: Complex{T}
    re::T
    im::T
end

# bits type
bitstype 128 Complex128 <: Complex{Float64}

# type alias
typealias TwoOf{T} (T,T)
\end{verbatim}
\end{singlespace}

\subsubsection{Constructors}

Composite types are applied as functions to construct instances.
The default constructor accepts values for each field as arguments.
Users may override the default constructor by writing method definitions
with the same name as the type inside the {\tt type} block. Inside the
{\tt type} block the identifier {\tt new} is bound to a pseudofunction
that actually constructs instances from field values. The constructor
for the {\tt Rational} type is a good example:

\vspace{-0.2in}
\begin{singlespace}
\begin{verbatim}
type Rational{T<:Integer} <: Real
    num::T
    den::T

    function Rational(num::T, den::T)
        if num == 0 && den == 0
            error("invalid rational: 0//0")
        end
        g = gcd(den, num)
        new(div(num, g), div(den, g))
    end
end
\end{verbatim}
\end{singlespace}

This allows {\tt Rational} to enforce representation as a fraction in
lowest terms.


\section{Generic Functions}

% talk about how this is a sneaky way to collect type information. since
% signatures specify dispatch behavior they are not redundant information.

The vast majority of Julia functions (in both the library and user programs)
are generic functions, meaning they contain multiple definitions or methods for
various combinations of argument types. When a generic function is applied,
the most specific definition that matches the run-time argument types is
invoked. Generic functions have appeared in several object systems in the past,
notably CLOS \cite{closoverview} and Dylan \cite{dylanlang}.
Julia is distinguished from these in that
it uses generic functions as its primary abstraction mechanism, putting it in
the company of research languages like Diesel \cite{dieselspec}. Aside
from being practical for highly polymorphic mathematical styles of programming,
as we will discuss, this design is satisfying also because it permits
expression of most of the popular patterns of object-oriented programming,
while leaving the core language with fewer distinct features.

Generic functions are a natural fit for mathematical programming. For example,
consider implementing exponentiation (the {\tt \^{}} operator in Julia).
This function
lends itself to multiple definitions, specializing on both arguments
separately: there might be one definition for two floating-point numbers that
calls a standard math library routine, one definition for the case where the
second argument is an integer, and separate definitions for the case where the
first argument is a matrix. In Julia these signatures would be written as
follows:

\begin{verbatim}
function ^(x::Float64, p::Float64)
function ^(x, p::Int)
function ^(x::Matrix, p)
\end{verbatim}

\subsection{Singleton Kinds}

A generic function's method table is effectively a dictionary where the keys
are types. This suggests that it should be just as easy to define or look up
methods with types themselves as with the types of values. Defining methods on
types directly is analogous to defining class methods in class-based object
systems. With multi-methods, definitions can be associated with combinations
of types, making it easy to represent properties not naturally owned by one
type.

To accomplish this, we introduce a special singleton kind {\tt Type\{T\}},
which contains the type {\tt T} as its only value.
%This is similar to the
%self-type pattern [cite], except along the type-of hierarchy rather than along
%the subtype-of hierarchy.
The result is a feature similar to {\tt eql}
specializers in CLOS, except only for types. An example use is defining
type traits:

\begin{verbatim}
typemax(::Type{Int64}) = 9223372036854775807
\end{verbatim}

This definition will be invoked by the call {\tt typemax(Int64)}. Note that
the name of a method argument can be omitted if it is not referenced.

Types are useful as method arguments in several other cases. One example is
file I/O, where a type can be used to specify what to read. The call
{\tt read(file,Int32)} reads a 4-byte integer and returns it as an {\tt Int32}
(a fact that the type inference process is able to discover). We find this
more elegant and convenient than systems where enums or special constants must
be used for this purpose, or where the type information is implicit
(e.g. through return-type overloading).

\subsection{Method Sorting and Ambiguity}
Methods are stored sorted by specificity, so the first matching method
(as determined by the subtype predicate) is always the correct one to invoke.
This means much of the dispatch logic is contained in the sorting process.
Comparing method signatures for specificity is not trivial. As one might
expect, the ``more specific''\footnote{Actually, ``not less specific'',
since specificity is a partial order.}
predicate is quite similar to the subtype
predicate, since a type that is a subtype of another is indeed more specific
than it. However, a few additional rules are necessary to capture the
intuitive concept of ``more specific''. In fact until this point
``more specific'' has had no formal meaning; its formal definition is
summarized as the disjunction of the following rules ($A$ is more specific
than $B$ if):

\pagebreak

\begin{enumerate}
\item $A$ is a subtype of $B$
\item $A$ is of the form {\tt T\{P\}} and $B$ is of the form {\tt S\{Q\}}, and
$T$ is a subtype of $S$ for some parameter values
\item The intersection of $A$ and $B$ is nonempty, more specific than $B$, and
not equal to $B$, and $B$ is not more specific than $A$
\item $A$ and $B$ are tuple types, $A$ ends in a vararg ({\tt ...}) type,
and $A$ would be more specific than $B$ if its vararg type were expanded to
give it the same number of elements as $B$
\end{enumerate}

Rule 2 means that declared subtypes are always more specific than their
declared supertypes regardless of type parameters. Rule 3 is mostly useful for
union types: if $A$ is {\tt Union(Int32,String)} and $B$ is {\tt Number}, $A$
should
be more specific than $B$ because their intersection ({\tt Int32}) is clearly
more specific than $B$. Rule 4 means that argument types are more important for
specificity than argument count; if $A$ is {\tt (Int32...)} and $B$ is
{\tt (Number, Number)} then $A$ is more specific.

Julia uses \emph{symmetric} multiple dispatch, which means all argument types
are equally important. Therefore, ambiguous signatures are possible.
For example, given {\tt foo(x::Number, y::Int)} and
{\tt foo(x::Int, y::Number)} it is not clear which method to call when both
arguments are integers. We detect ambiguities when a method is added, by
looking for a pair of signatures with a non-empty intersection where neither
one is more specific than the other. A warning message is displayed for each
ambiguity, showing the user the computed type intersection so it is clear what
definition is missing. For example:

\begin{verbatim}
Warning: New definition foo(Int,Number) is ambiguous with foo(Number,Int).
        Make sure foo(Int,Int) is defined first.
\end{verbatim}


\section{Intrinsic Functions}

The run-time system contains a small number of primitive functions for tasks
like determining the type of a value, accessing fields of composite types, and
constructing values of each of the supported kinds of concrete types. There are
also arithmetic intrinsics corresponding to machine-level operations like
fixed-width integer addition, bit shifts, etc. These intrinsic functions are
implemented only in the code generator and do not have callable entry points.
They operate on bit strings, which are not first class values but can be
converted to and from Julia bits types via boxing and unboxing operations.

In our implementation, the core system also provides functions for constructing
arrays and accessing array elements. Although this is not strictly necessary,
we did not want to expose unsafe memory operations (e.g. load and store
primitives) in the language. In an earlier implementation, the core system
provided a bounds-checked {\tt Buffer} abstraction, but having both this type
and the user-level {\tt Array} type proved inconvenient and confusing.


\section{Design Limitations}

In our design, type information always flows along with values, in the
forward control flow direction. This prevents us from doing certain tricks
that static type systems are capable of, such as return-type overloading.
Return-type overloading requires a robust notion of the type of a value
\emph{context}---the type expected or required of some term---in order to
select code on that basis. There are other cases where ``backwards'' type
flow might be desirable, such as determining the type of a container based
on the type of a value stored into it at a later program point. It may be
possible to get around this limitation in the future using inversion of
control---passing a function argument whose result type has already been
inferred, and using that type to construct a container before elements are
computed.

Modularity is a perennial difficulty with multiple dispatch, as any
function may apply to any type, and there is no point where functions or
types are closed to future definitions. Thus at the moment Julia is
essentially a whole-program compiler. We plan to implement a module system
that will at least allow code to control which name bindings and definitions
it sees. Such modules could be separately compiled to the extent that
programmers are willing to ask for their definitions to be ``closed''.

Lastly, at this time Julia uses a bit more memory than we would prefer.
Our compiler data structures, type information, and generated native code
take up more space than the compact bytecode representations used by many
dynamic languages.


%\section{Extra Features}

%symmetric coroutines

%macros

%ccall
